\section*{MATH1231/1241 Algebra S2 2008 Test 1 Version 3B}
\begin{enumerate}
    % Q1: subspace
    \item We have
        $$S = \left\{ \x \in \R^4 \left|~ \x = \lambda (2, 0, 1, -1)^T
        \textrm{ for some } \lambda \in \R \right. \right\}$$

        \begin{itemize}
            \item
                Setting $\lambda = 0 \in \R$, we get
                $$\v{0} \in S$$
                and thus S is not empty.

            \item
                Suppose $\x, \y \in S$. Then,
                \begin{align*}
                    &\left\{
                        \begin{matrix}
                            \x = \lambda (2, 0, 1, -1)^T, ~ \lambda \in \R \\
                            \y = \mu (2, 0, 1, -1)^T, ~ \mu \in \R
                        \end{matrix}
                    \right. \\
                    \implies&
                        (\x+\y) = (\lambda + \mu) (2, 0, 1, -1)^T \\
                    \implies&
                        (\x+\y) \in S ~ \textrm{ since } (\lambda+\mu) \in \R
                \end{align*}

                Thus $S$ is closed under vector addition.

                Further, suppose that $\alpha \in \R$. Then,
                \begin{align*}
                    \alpha \x &= \alpha (\lambda (2, 0, 1, -1)^T) \\
                    \alpha \x &= (\alpha \lambda) (2, 0, 1, -1)^T \\
                    \implies \alpha \x &\in S
                \end{align*}

                So $S$ is closed under scalar multiplication.
        \end{itemize}

        It follows that $S$ is a subspace of $\R^4$ by the Subspace Theorem.

    % Q2: spans
    \item
        \begin{enumerate}
            \item 
            Suppose that $\v{w} = (a_1, a_2, a_3)^T \in \R^3$.
            Let $S = \{\v{v_1}, \v{v_2}, \v{v_3}, \v{v_4}\}$.

            Then, $S$ spans $\R^3$ if and only if $\v{w} \in \textrm{span}(S)$
            for any choice of $\v{w}$.

            This is equivalent to existence of a solution for the system of
            linear equations represented by the following augmented matrix:

            $$\begin{amatrix}{4}
                -1 &  3 &  1 &  2 & a_1 \\
                -2 &  7 &  3 &  3 & a_2 \\
                 2 & -1 &  3 & -9 & a_3 \\
            \end{amatrix}$$

            Row-reducing, we obtain:

            \begin{math}
                \xrightarrow[R3 = R3 + 2R1]{R2 = R2 - 2R1}
                \begin{amatrix}{4}
                    -1 &  3 &  1 &  2 & a_1 \\
                     0 &  1 &  1 & -1 & a_2 - 2a_1 \\
                     0 &  5 &  5 & -5 & a_3 + 2a_1 \\
                \end{amatrix}
                \xrightarrow{R3 = R3 - 5R2}
                \begin{amatrix}{4}
                    -1 &  3 &  1 &  2 & a_1 \\
                     0 &  1 &  1 & -1 & a_2 - 2a_1 \\
                     0 &  0 &  0 &  0 & 12a_1 - 5a_2 + a_3 \\
                \end{amatrix}
            \end{math} \\[5mm]

            The right-hand column is leading.
            Thus, the system has a solution if and only if
            $$12a_1 - 5a_2 + a_3 = 0$$

            That is also the condition which must be satisfied for $\v{w}$
            to be in the span of $S$.

            Thus $S$ does not span all of $\R^3$.

            \item
                Considering $\v{u}$,
                $$12(1) - 5(0) + (-5) = 7 \neq 0$$
                Thus $\v{u} \notin \textrm{span}(S)$.
        \end{enumerate}

    % Q3: linear independence
    \item
        The vectors are linearly independent if and only if the system
        of linear equations, represented by the following augmented matrix,
        has only one (trivial) solution.

        $$\begin{amatrix}{3}
             1 & -4 &  3 & 0 \\
            -3 &  7 & -2 & 0 \\
            -1 & -1 &  4 & 0 \\
        \end{amatrix}$$

        Row-reducing, we get:

        \begin{math}
            \xrightarrow[R3 = R3 + R1]{R2 = R2 + 3R1}
            \begin{amatrix}{3}
                 1 & -4 &  3 & 0 \\
                 0 & -5 &  7 & 0 \\
                 0 & -5 &  7 & 0 \\
            \end{amatrix}
            \xrightarrow{R3 = R3 - R2}
            \begin{amatrix}{3}
                 1 & -4 &  3 & 0 \\
                 0 & -5 &  7 & 0 \\
                 0 &  0 &  0 & 0 \\
            \end{amatrix}
        \end{math} \\[5mm]

        Third column is non-leading, and therefore the system has
        infinitely many solutions.

        Therefore, the vectors are linearly dependent.
\end{enumerate}
