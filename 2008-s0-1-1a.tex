\section*{MATH1231/1241 Algebra S2 2007 Test 1 Version 1B}
\begin{enumerate}
    % Q1: matrix subspace
    \item
        We have $S = \left\{ \left. \begin{pmatrix}
            a_{11} & a_{12} \\
            a_{21} & a_{22} \\
        \end{pmatrix} \right| a_{11} + a_{12} = 3a_{22} \right\}$.

        \begin{itemize}
            \item
                Clearly, $\v{0} \in S$:
                $$0 + 0 = 3(0)$$
                and so $S$ is not empty.

            \item
                Suppose $A, B \in S$. Then,
                \begin{align*}
                    &\left\{\begin{matrix}
                        [A]_{11} + [A]_{12} = 3[A]_{22} \\
                        [B]_{11} + [B]_{12} = 3[B]_{22}
                    \end{matrix}\right. \\
                    \implies
                    &[A+B]_{11} + [A+B]_{12} = 3[A+B]_{22} \\
                    \implies
                    &(A + B) \in S
                \end{align*}

                And so $S$ is closed under vector addition.

            \item
                Suppose $A \in S$ and $\lambda \in \R$. Then,
                \begin{align*}
                    [A]_{11} + [A]_{12} &= 3[A]_{22} \\
                    \lambda[A]_{11} + \lambda[A]_{12} &= \lambda[A]_{22} \\
                    [\lambda A]_{11} +  [\lambda A]_{12} &=  [\lambda A]_{22} \\
                    \implies
                    \lambda A \in S
                \end{align*}

                And so $S$ is closed under scalar multiplication.
        \end{itemize}

        It follows that set $S$ is a subspace of the vector
        space $\M_{2,2}(\R)$ by the Subspace Theorem.

        % Q2: linear (in)dependence
    \item
        \begin{enumerate}
            \item
                Vectors $\v{v_1}, \v{v_1}, \dots, \v{v_n} \in V$ are
                linearly independent if there exist scalars
                $\lambda_1, \dots, \lambda_n$, not all zero,
                such that the following equation holds:
                $$\lambda_1\v{v_1} + \lambda_2\v{v_2} + \dots + \lambda_n\v{v_n} = \v{0}$$
            \item
                First, we rewrite the matrices in vector form. We need
                to establish if the following vectors are linearly
                dependent:
                $$
                \v{A'_1} = \begin{pmatrix}1 \\ 2 \\ 2 \\ 1\end{pmatrix};~
                \v{A'_2} = \begin{pmatrix}0 \\ 1 \\ 1 \\ 0\end{pmatrix};~
                \v{A'_3} = \begin{pmatrix}2 \\ 1 \\ 2 \\ 1\end{pmatrix}
                $$

                They are linearly dependent if and only if the system of
                linear equations represented by the following augmented
                matrix has more than one solution.
                $$
                    \begin{amatrix}{3}
                        1 & 0 & 2 & 0 \\
                        2 & 1 & 1 & 0 \\
                        2 & 1 & 2 & 0 \\
                        1 & 0 & 1 & 0 \\
                    \end{amatrix}
                $$

                Row-reducing, we obtain

                \begin{math}
                    \xrightarrow[R4 = R4 - R1]{R2 = R2 - 2R1;~ R3 = R3 - 2R1}
                    \begin{amatrix}{3}
                        1 & 0 & 2  & 0 \\
                        0 & 1 & -3 & 0 \\
                        0 & 1 & -2 & 0 \\
                        0 & 0 & -1 & 0 \\
                    \end{amatrix}
                    \xrightarrow{R3 = R3 - R2}
                    \begin{amatrix}{3}
                        1 & 0 & 2  & 0 \\
                        0 & 1 & -3 & 0 \\
                        0 & 0 & 1  & 0 \\
                        0 & 0 & -1 & 0 \\
                    \end{amatrix} \\
                    \xrightarrow{R4 = R4 + R3}
                    \begin{amatrix}{3}
                        1 & 0 & 2  & 0 \\
                        0 & 1 & -3 & 0 \\
                        0 & 0 &  1 & 0 \\
                        0 & 0 & 0  & 0 \\
                    \end{amatrix}
                \end{math}

                The last column is non-leading, and there are no other
                non-leading columns, and therefore the system has
                exactly one solution.

                Therefore the vectors, and hence the corresponding
                matrices, are linearly independent.
        \end{enumerate}

        % Q3: basis
    \item
        \begin{enumerate}
            \item
                To find a basis for $W$, we consider the leading columns
                of a matrix which columns are the vectors
                $\v{v_1}, \dots, \v{v_4}$ when reduced to REF. \\[5mm]

                \begin{math}
                    \begin{pmatrix}
                        1 & 2 & -1 & 1 \\
                        3 & 7 &  0 & 1 \\
                        1 & 1 & -4 & 3 \\
                    \end{pmatrix}
                    \xrightarrow[R3 = R3 - R1]{R2 = R2 - 3R1}
                    \begin{pmatrix}
                        1 &  2 & -1 &  1 \\
                        0 &  1 &  3 & -2 \\
                        0 & -1 & -3 &  2 \\
                    \end{pmatrix}
                    \xrightarrow{R3 = R3 + R2}
                    \begin{pmatrix}
                        1 &  2 & -1 &  1 \\
                        0 &  1 &  3 & -2 \\
                        0 &  0 &  0 &  0 \\
                    \end{pmatrix}
                \end{math} \\[5mm]

                The first and second columns are leading, which
                correspond to the vectors $\v{v_1}, \v{v_2}$.
                Therefore, the vectors $\v{v_1}, \v{v_2}$ form the basis
                for $W$.

            \item
                Such vector is $\v{v_4}$.

                We want to solve
                $$\lambda_1 \v{v_1} + \lambda_2 \v{v_2} + \lambda_4 \v{v_4} = \v{0}$$

                Forming an augmented matrix and row-reducing, we obtain: \\[5mm]

                \begin{math}
                    \begin{amatrix}{3}
                        1 & 2 &  1 & 0 \\
                        3 & 7 &  1 & 0 \\
                        1 & 1 &  3 & 0 \\
                    \end{amatrix}
                    \xrightarrow[R3 = R3 - R1]{R2 = R2 - 3R1}
                    \begin{amatrix}{3}
                        1 &  2 &  1 & 0 \\
                        0 &  1 & -2 & 0 \\
                        0 & -1 &  2 & 0 \\
                    \end{amatrix}
                    \xrightarrow{R3 = R3 + R2}
                    \begin{amatrix}{3}
                        1 &  2 &  1 & 0 \\
                        0 &  1 & -2 & 0 \\
                        0 &  0 &  0 & 0 \\
                    \end{amatrix}
                \end{math} \\[5mm]

                Third column is non-leading. If we set $\lambda_3 = 1$
                and back-substitute, we obtain:
                \begin{align*}
                    \lambda_2 - 2\lambda_3 = 0 &\implies
                        \lambda_2 = 2\lambda_3 = 2 \\
                    \lambda_1 + 2\lambda_2 + \lambda_3 = 0 &\implies
                        \lambda_1 = -5\lambda_3 = -5
                \end{align*}
                and thus obtain the required result,
                $$-5\v{v_1} + 2\v{v_2} + \v{v_4} = \v{0}$$
                that is,
                $$5\v{v_1} - 2\v{v_2} = \v{v_4}$$
        \end{enumerate}
\end{enumerate}
